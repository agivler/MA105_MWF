
\documentclass[fleqn, letterpaper]{article}
\usepackage{amssymb,amsmath}
\usepackage[left=2cm,top=2cm,right=2cm,nohead,nofoot]{geometry}
\usepackage{verbatim}
\usepackage{ulem}
\usepackage{graphicx}
\usepackage{lmodern}
\usepackage{tabu}
\usepackage{url}
\usepackage{enumerate}

%\title{MA 105 Chapter 1 Project}

\begin{document}
%\maketitle
\noindent

\begin{center}\textbf{MA 105 Ch. 2, 3 Practice}\end{center}

\noindent
{\bfseries Directions:} Below are 5 different data sets. You will be practicing the data analysis skills you have learned in chapters 2 and 3. Complete each question for the data sets indicated.

\begin{enumerate}
\item Construct a histogram of the data, using 5 classes. (sets 2 and 3)
\item Describe the distribution. What shape best describes the distribution? (sets 2 and 3)
\item Calculate the mean and median (all sets). Are the mean and median of the same data set relatively similar? (sets 2 and 3 only) If not, which is larger, and would you expect that to be true for other distributions of this type?
\item Calculate the range, standard deviation, and variance. (all sets)
\item Identify any potential outliers. Test to see if any of these values are unusual. (all sets)
\item Compare and contrast the data set 1 and 3. Does it make sense to directly compare the means and the standard deviations? Why or why not?
\item List at least 2 reasonable ways of displaying the data or a summary of the data (be specific). List at least one type of graph or table that would not be appropriate or useful for that data and explain why. (sets 4 and 5).
\end{enumerate}

\noindent
{\bfseries Data Set 1}


\noindent
This data set consists of the numbers of years that popes (since 1690) lived after their election or coronation. \\


\noindent
2, 9, 21, 3, 6, 10, 18, 11, 6, 25, 23, 6, 2, 15, 32, 25, 11, 8, 17, 19, 5, 15, 0, 26\\
%This data set consists of brain volumes in cm$^3$

%1005, 963, 1035, 1027, 1281, 1272, 1051, 1079, 1034, 1070, 1173, 1079, 1067, 1104, 1347, 1439, 1029, 1100, 1204, 1160

\noindent
{\bfseries Data Set 2}


\noindent
This data set consists of the numbers of years that British monarchs (since 1690) lived after their election or coronation. \\


\noindent
17, 6, 13, 12, 13, 33, 59, 10, 7, 63, 9, 25, 36, 15

\noindent
{\bfseries Data Set 3}


\noindent
This data set consists of the times (in minutes) spent on hygiene and grooming in the morning from randomly selected subjects.\\


\noindent
0, 5, 12, 15, 15, 20, 22, 24, 25, 25, 25, 27, 27, 28, 30, 30, 35, 35, 40, 45\\

\noindent
{\bfseries Data Set 4}

\noindent
This data set consists of the daily weather information (date, highest temperature, lowest temperature, precipitation amount) for Lexington, VA from 1971 to 2000 (see http://www.sercc.com/cgi-bin/sercc/cliMAIN.pl?va4876 for the complete data set).\\

\noindent
{\bfseries Data Set 5}

\noindent
This data set consists of voter information from exit polls from the recent election cycle and includes who the voter voted for and the voter's age, gender, race, level of education, and household income.
\end{document}

